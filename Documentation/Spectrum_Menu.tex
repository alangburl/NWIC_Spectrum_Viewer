\section{View Spectrum Data}
	This drop down has options that come directly from the data loaded into the main interface. 

		\subsection{View Energies}
			This takes a set of energies, in MeV, separated by commas to draw vertical lines on the graph. These lines will be drawn as a dotted red line.
		\subsection{Change Zoom Location}
			Changes the minimum and maximum energies shown on the graph. Much like a zoom feature. 
		\subsection{Calibration Energies}
			Takes a set of values, separated by commas to draw vertical lines representing calibration energies. These will appear as solid black lines. 
		\subsection{Count Rates}
			Will display the count rate of all loaded spectrum in counts per second. If a time of 1 second has been entered for the accumulation time, the value displayed will be the total number of counts in the total spectrum.
		\subsection{ROI Uncertainties}
		This will compute the gross count rate, net count rate, and the associate uncertainties for each in a specified region of interest. The gross count uncertainties are found using $\sigma=\sqrt{N}$, where $\sigma$ is the standard deviation and $N$ is the number of counts in the region of interest. The net uncertainty is found using equation \ref{eq:net_uncer}, where $\sigma_F$ and $\sigma_B$ are foreground and background standard deviations respectively.
		\begin{equation}
			\sigma_{net}=\sqrt{\sigma_F^2+\sigma_B^2}
			\label{eq:net_uncer}
		\end{equation}
		
		\subsection{Energy Resolution}
			Once a spectrum has been plotted and is located in the Plotted Spectrum section, the option to view the energy resolution will be enabled. This option uses a peak finding algorithm to analyze and determine the peaks. The energy resoution is calculated as the full width half maximum of the identified peak divided by the peak centroid location. This is calculated and displayed for each peak found in the spectrum. The energy resolution is displayed next to a vertical line as a percentage. 